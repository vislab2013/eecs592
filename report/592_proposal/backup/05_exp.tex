\section{Experimental Design}
\label{sec:5}

We aim to evaluate our algorithm on highly-cluttered indoor images with human presented. Since there is no available dataset satisfies our need, we will propose a new dataset that we call the \textit{indoor-human-activity} dataset for our evaluation. The \textit{indoor-human-activity} dataset will contains 911 images, and is composed of five human activity classes: dancing (187), having dinner (183), talking (193), washing dishes (183), and watching TV (165). Fig. \ref{fig:f5-1} shows sample images of the dataset. We will obtain a thorough annotation for objects, humans, and the layout for the dataset.

We plan to apply poselet detector \cite{Bourdev_ICCV2009,Bourdev_ECCV2010} to provide candidate human bounding boxes. To handle different human poses, we classify each candidate detection into two poses: standing and sitting, and assign different 3D CAD models for each pose. The classifier is trained by LIBSVM \cite{CC01a}. We adopt 50 images from each activity class for training the human pose classifier, and use the rest for evaluating the layout estimation. We consider the predicted human bounding boxes that have more than 50\% overlap with certain ground-truth bounding boxes to be our training data. We use the weighted poselet activation vector and the height ratio between full body and torso as the feature to train the SVM. We also plan to show some result of object detection on the dataset, and argue that objects can not be detected robustly.

We first evaluate the accuracy of vanishing point estimation (Sec \ref{sec:5-1}). In Sec \ref{sec:5-2}, we first show better estimated vanishing points can generate better candidate layout hypotheses, and then we analyze the contribution of layout estimation error by input vanishing points.


\subsection{Vanishing Point Estimation}
\label{sec:5-1}
Given three orthogonal vanishing points, we can calculate the pitch, yaw, roll angles, and the focal length of the camera. We adopt these values as the metrics of our evaluation. 

We compare the results of following methods.
\begin{itemize}
  \item Hedau's method \cite{Hedau_ICCV2009}
  \item Our partial method (w/o human): only $\Psi(I,S)$
  \item Our full method with GT human bounding boxes
  \item Our full method with poselet detections
  %\item Our full method (apply prior on scene parameters)
\end{itemize}

\subsection{Room Layout Estimation}
\label{sec:5-2}
First we analyze the candidate layouts generated by \cite{Hedau_ICCV2009} given three orthogonal vanishing points. We show that by improving vanishing point estimation, the best candidate layout can achieved lower pixel error.

Next, we compare the layout estimation results (in pixel error) of following methods.
\begin{itemize}
  \item Using VPs given by Hedau's method \cite{Hedau_ICCV2009}
  \item Using ground-truth VPs
  \item Using VPs given by our full method
\end{itemize}

\begin{figure*}[t]
  \centering
  \begin{subfigure}[b]{0.20\textwidth}
    \centering
    \begin{subfigure}[b]{0.23\textwidth}
      \centering
      \centerline{\includegraphics[height=2.5\textwidth]{figure/sample_img/dc1.jpg}}
    \end{subfigure} \\
    ~\vspace{-3.0mm}\\
    \begin{subfigure}[b]{0.23\textwidth}
      \centering
      \centerline{\includegraphics[height=2.5\textwidth]{figure/sample_img/dc2.jpg}}
    \end{subfigure} \\
    ~\vspace{-3.0mm}\\
    \begin{subfigure}[b]{0.23\textwidth}
      \centering
      \centerline{\includegraphics[height=2.5\textwidth]{figure/sample_img/dc3.jpg}}
    \end{subfigure} \\
    ~\vspace{-3.0mm}\\
    \begin{subfigure}[b]{0.23\textwidth}
      \centering
      \centerline{\includegraphics[height=2.5\textwidth]{figure/sample_img/dc4.jpg}}
    \end{subfigure} \\
    \caption{dancing}
  \end{subfigure}
  ~\hspace{-3mm}
  \begin{subfigure}[b]{0.20\textwidth}
    \centering
    \begin{subfigure}[b]{0.23\textwidth}
      \centering
      \centerline{\includegraphics[height=2.5\textwidth]{figure/sample_img/hd1.jpg}}
    \end{subfigure} \\
    ~\vspace{-3.0mm}\\
    \begin{subfigure}[b]{0.23\textwidth}
      \centering
      \centerline{\includegraphics[height=2.5\textwidth]{figure/sample_img/hd2.jpg}}
    \end{subfigure} \\
    ~\vspace{-3.0mm}\\
    \begin{subfigure}[b]{0.23\textwidth}
      \centering
      \centerline{\includegraphics[height=2.5\textwidth]{figure/sample_img/hd3.jpg}}
    \end{subfigure} \\
    ~\vspace{-3.0mm}\\
    \begin{subfigure}[b]{0.23\textwidth}
      \centering
      \centerline{\includegraphics[height=2.5\textwidth]{figure/sample_img/hd4.jpg}}
    \end{subfigure} \\
    \caption{having dinner}
  \end{subfigure}
  ~\hspace{-3mm}
  \begin{subfigure}[b]{0.20\textwidth}
    \centering
    \begin{subfigure}[b]{0.23\textwidth}
      \centering
      \centerline{\includegraphics[height=2.5\textwidth]{figure/sample_img/tk1.jpg}}
    \end{subfigure} \\
    ~\vspace{-3.0mm}\\
    \begin{subfigure}[b]{0.23\textwidth}
      \centering
      \centerline{\includegraphics[height=2.5\textwidth]{figure/sample_img/tk2.jpg}}
    \end{subfigure} \\
    ~\vspace{-3.0mm}\\
    \begin{subfigure}[b]{0.23\textwidth}
      \centering
      \centerline{\includegraphics[height=2.5\textwidth]{figure/sample_img/tk3.jpg}}
    \end{subfigure} \\
    ~\vspace{-3.0mm}\\
    \begin{subfigure}[b]{0.23\textwidth}
      \centering
      \centerline{\includegraphics[height=2.5\textwidth]{figure/sample_img/tk4.jpg}}
    \end{subfigure} \\
    \caption{talking}
  \end{subfigure}
  ~\hspace{-3mm}
  \begin{subfigure}[b]{0.20\textwidth}
    \centering
    \begin{subfigure}[b]{0.23\textwidth}
      \centering
      \centerline{\includegraphics[height=2.5\textwidth]{figure/sample_img/wd1.jpg}}
    \end{subfigure} \\
    ~\vspace{-3.0mm}\\
    \begin{subfigure}[b]{0.23\textwidth}
      \centering
      \centerline{\includegraphics[height=2.5\textwidth]{figure/sample_img/wd2.jpg}}
    \end{subfigure} \\
    ~\vspace{-3.0mm}\\
    \begin{subfigure}[b]{0.23\textwidth}
      \centering
      \centerline{\includegraphics[height=2.5\textwidth]{figure/sample_img/wd3.jpg}}
    \end{subfigure} \\
    ~\vspace{-3.0mm}\\
    \begin{subfigure}[b]{0.23\textwidth}
      \centering
      \centerline{\includegraphics[height=2.5\textwidth]{figure/sample_img/wd4.jpg}}
    \end{subfigure} \\
    \caption{washing dishes}
  \end{subfigure}
  ~\hspace{-3mm}
  \begin{subfigure}[b]{0.20\textwidth}
    \centering
    \begin{subfigure}[b]{0.23\textwidth}
      \centering
      \centerline{\includegraphics[height=2.5\textwidth]{figure/sample_img/wt1.jpg}}
    \end{subfigure} \\
    ~\vspace{-3.0mm}\\
    \begin{subfigure}[b]{0.23\textwidth}
      \centering
      \centerline{\includegraphics[height=2.5\textwidth]{figure/sample_img/wt2.jpg}}
    \end{subfigure} \\
    ~\vspace{-3.0mm}\\
    \begin{subfigure}[b]{0.23\textwidth}
      \centering
      \centerline{\includegraphics[height=2.5\textwidth]{figure/sample_img/wt3.jpg}}
    \end{subfigure} \\
    ~\vspace{-3.0mm}\\
    \begin{subfigure}[b]{0.23\textwidth}
      \centering
      \centerline{\includegraphics[height=2.5\textwidth]{figure/sample_img/wt4.jpg}}
    \end{subfigure} \\
    \caption{watching TV}
  \end{subfigure}
  \caption{Sample images from our collected \textit{indoor-human-activity} dataset.}
  \label{fig:f5-1}
\end{figure*}